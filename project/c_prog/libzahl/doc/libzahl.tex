\documentclass[11pt,b5paper,openright,fleqn]{book}
\special{papersize=176mm,250mm}
\usepackage[utf8]{inputenc}
\usepackage[T1]{fontenc}
\usepackage{algorithmic, algorithm, colonequals, alltt}
\usepackage{amsmath, amssymb, mathtools, MnSymbol, mathrsfs, esvect, wasysym}
\usepackage{tipa, color, graphicx}
\usepackage{shorttoc, minitoc}
\usepackage{enumitem}
\usepackage[english]{babel}
\selectlanguage{english}
\definecolor{linkcolour}{RGB}{112, 0, 0}
\definecolor{urlcolour}{RGB}{0, 0, 112}
\usepackage[unicode,pdfencoding=auto]{hyperref}
\hypersetup{
  pdfborder={0 0 0},
  colorlinks=true,
  linkcolor=linkcolour,
  urlcolor=urlcolour,
  linktoc=all,
  pdfsubject={Computer science},
  pdfauthor={Mattias Andrée},
  pdftitle={libzahl},
  pdfkeywords={libzahl, big integer, big number, bigint, bignum, multiple-precision, arbitrary precision}
}
\hypersetup{linktocpage}
\usepackage{makeidx}
\makeindex
\usepackage{geometry}
\geometry{margin=1in}
\usepackage{microtype}
\DisableLigatures{encoding = *, family = *} % NB! disables -- and ---
% I really dislike fi- and ff-ligatures, just like look so wrong.
\frenchspacing % i.e. non-American spacing: i.e. no extra space after sentences,
               % this also means that periods do not have to be context-marked.

\newcommand{\chapref}[1]{\hyperref[#1]{Chapter~\ref*{#1} [\nameref*{#1}], page \pageref*{#1}}}
\newcommand{\secref}[1]{\hyperref[#1]{Section~\ref*{#1} [\nameref*{#1}], page \pageref*{#1}}}
\newcommand{\appxref}[1]{\hyperref[#1]{Appendix~\ref*{#1} [\nameref*{#1}], page \pageref*{#1}}}
\newcommand{\pchapref}[1]{(see \hyperref[#1]{Chapter~\ref*{#1} [\nameref*{#1}], page \pageref*{#1}})}
\newcommand{\psecref}[1]{(see \hyperref[#1]{Section~\ref*{#1} [\nameref*{#1}], page \pageref*{#1}})}
\newcommand{\pappxref}[1]{(see \hyperref[#1]{Appendix~\ref*{#1} [\nameref*{#1}], page \pageref*{#1}})}
\definecolor{c}{rgb}{0.45, 0.45, 0.45}

\begin{document}

\frontmatter

\title{{\Huge \bf libzahl version 1.1}}
\author{}
\date{\vspace{3in}}
\maketitle

\thispagestyle{empty}
\null
\vfill
\noindent
Copyright \copyright{} 2016 $~$  Mattias Andrée $\langle$\href{mailto:maandree@kth.se}{\texttt{maandree@kth.se}}$\rangle$
\vspace{1ex}

\noindent
{\small
Permission to use, copy, modify, and/or distribute this document for any
purpose with or without fee is hereby granted, provided that the above
copyright notice and this permission notice appear in all copies.}
\newpage


% Conventionally, most words in a title in English should start with
% uppercase. I believe that this is inconsistent stupidity, pardon my
% Klatchian. There is not consensus of which words should not start
% with lowercase or even if any shall start with lowercase. There is
% also no consensus on how long the title should be before only the
% first word should start with uppercase. It is only generally (but
% not always) agreed that most words should start with uppercase and
% when the title is too long only the first word start with uppercase.
% I believe that is is better to stick with the Swedish convention:
% It should look just like a sentience except it may not end with a
% period unless that is part of an ellipsis or an abbreviation.
% I would also like to use straight apostrophes, like in French, (and
% reserve the curved ones for quotes), but that is just too painful in
% LaTeX, so I will only be do so for French words. Most style guides
% for English will be followed. They will only be broken if they are
% stupid or inferior. For example, I will never write ‘CPU's’ for
% plural of CPU — that's just stupid, — only for genitive, nor
% will I write ‘CPUs’ for plural of CPU, because it is inferior to
% ‘CPU:s’.


\shorttoc{Short contents}{0}
\setcounter{tocdepth}{2}
\dominitoc
\tableofcontents


\mainmatter

\input doc/what-is-libzahl.tex
\input doc/libzahls-design.tex
\input doc/get-started.tex
\input doc/miscellaneous.tex
\input doc/arithmetic.tex
\input doc/bit-operations.tex
\input doc/number-theory.tex
\input doc/random-numbers.tex
\input doc/not-implemented.tex
\input doc/exercises.tex


\appendix


\backmatter

\addcontentsline{toc}{chapter}{Index}
\printindex

\end{document}
